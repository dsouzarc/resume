% LaTeX resume using res.cls
\documentclass[line,margin]{res} 
%\usepackage{helvetica} % uses helvetica postscript font (download helvetica.sty)
%\usepackage{newcent}   % uses new century schoolbook postscript font 

\begin{document}

\name{\textbf{Ryan D'souza}}
% \address used twice to have two lines of address
\address{23 Silvers Lane, Cranbury, NJ, 08512. (609) 915 - 4930}
\address{dsouzarc@gmail.com, linkedin.com/in/dsouzarc \hspace{20ex}}
\begin{resume}
\title{DJJD}
 
\section{DESCRIPTION}An aspiring 17 year old looking for a career in Computer Science, Finance, and Economics

\section{Technologies}	Java: GUI/SWING, AWT, Database, SQL, Network (POST I\& GET), I/O, MultiThreading, Applets

Android: Database/SQL, Animations, MultiThreading I\& Asynchronous Tasks, Networking, JSON Requests/Parsing, Notifications, Google Glass

Key Words: Facebook API, ZXING Library (Image Processing), Bloomberg API, C++, AppleScript, LaTex, Terminal, VBA, Mac OSX, Linux, Bash

\section{PROJECTS} {\sl Stock Calculator} \hfill Java Desktop App and Android Application \\
                Uses a custom algorithm to determine whether a stock's price will increase in the future.
                 \begin{itemize}  \itemsep -2pt % reduce space between items
                 \item The algorithm downloads and takes into account the company's P/E and Beta Ratio, Income Statement, Balance Sheet, Cash Flow, and historical trading data
                \item Uses Data Parsing to display the stock's live quote, as opposed to other apps that only give 15-minute delayed quotes
                \end{itemize}
 
                {\sl App Searcher} \hfill            Android Application \\
                Google Play's fastest Android equivalent of iOS's built-in application searcher.
                 \begin{itemize}  \itemsep -2pt %reduce space between items
                 \item Uses optimized algorithms and multithreading to load the user's list of installed apps four-five seconds faster then the competition's current load time. Speed is important when you want to open an app quickly.
                 \end{itemize} 
                 
                 \begin{itemize}  \itemsep -2pt 
                 \item Memory Efficient: Uses 80\% less memory then its competition
                 \end{itemize}
                 
                 \begin{itemize}  \itemsep -2pt 
                 \item App begins running on system boot and can be opened from either the Notification Bar or Chat Head. Runs silently and quietly in the Notification Bar, just click on it to open. Chat Head drawn on screen can be moved anywhere in the screen and clicking on it will immediately open AppSearcher.
                 \end{itemize}
                {\sl QEvent Share} \hfill Android Application\\
                Streamlines event creation and sharing.
                  \begin{itemize} \itemsep -2pt 
                   \item Shows all events the user has for the week in a week view beginning on the current date, as opposed to beginning on Sunday like all other calendar apps
                   \end{itemize} 
                   
                   \begin{itemize} \itemsep -2pt 
                   \item Easily add new events by only clicking on the date title of the event's date and inputting the time
                   \end{itemize}
                   
                   \begin{itemize} \itemsep -2pt 
                   \item Unique feature to share the event using QRCodes. Simply click on an already created event (or newly created event) and the option to share it via QR Code pops up. On the WeekView default screen, there is the option to Import a QRCode. When the QRCode is scanned, the event information is showed to the user who then can edit the event before saving it.
                   \end{itemize}
                   
                   {\sl PHS PowerSchool} \hfill Android Application \\
                   Application that streamlines a repetitive process for all PowerSchool users
                   \begin{itemize} \itemsep -2pt
                   \item 
                   Saves the student's Username and Password so that when they open the app, it automatically logs them into PowerSchool and shows their current Grades. This reduces the need for the student to go to the PowerSchool website, click Student Access, enter their Username and Password, click Submit, and then click on the Current Grades tab.
                   \end{itemize}
                   
                   \begin{itemize} \itemsep -2pt                   
                   \item App also has a GPA calculator that takes into account the current year’s grades when calculating the Weighted and Unweighted GPA
                   \end{itemize}
                   
                   \begin{itemize} \itemsep -2pt
                   \item Interesting feature: Click on assignment to add it to a list of assignments which is then copied to the ClipBoard. Convenient for students writing a ToDo list and need multiple assignments' titles, dates, and grades
                   \end{itemize}
\end{resume}
\end{document}







